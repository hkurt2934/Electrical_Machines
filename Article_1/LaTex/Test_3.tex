\documentclass{article}
\usepackage{amsmath}
\usepackage{graphicx}
\usepackage{hyperref}
\usepackage{geometry}
\geometry{margin=1in}

\begin{document}


\begin{abstract}
This paper proposes a novel regenerative braking control strategy for electric vehicles (EVs) driven by permanent magnet synchronous motors (PMSMs). The proposed method utilizes a cascaded bidirectional buck-boost converter in conjunction with a dual-mode fuzzy logic controller to maximize energy recovery and ensure safe deceleration across varying operating conditions. Unlike conventional braking schemes, the dual-mode approach dynamically switches between voltage-controlled and current-controlled modes based on motor back-electromotive force (EMF), battery state of charge (SOC), and vehicle speed. This enables effective braking even at low speeds where traditional regenerative strategies are ineffective. Simulation results using MATLAB/Simulink demonstrate the efficacy of the proposed method, showing improved energy recovery and braking responsiveness compared to fixed-mode approaches.
\end{abstract}

\title{Advanced Dual-Mode Regenerative Braking Control for PMSM-Based Electric Vehicles Using Cascaded Buck-Boost Converter and Fuzzy Logic}

\maketitle

\section{Introduction}
The rise of electric vehicles (EVs) has placed increasing importance on energy efficiency and battery longevity. Regenerative braking (RB) has emerged as a key technique for improving energy utilization by recovering kinetic energy during deceleration and storing it back into the battery. However, conventional RB methods are often limited in low-speed scenarios due to inadequate back-EMF generation.

Permanent magnet synchronous motors (PMSMs) are widely used in modern EVs due to their high efficiency, compact size, and excellent torque characteristics. Nevertheless, integrating effective regenerative braking with PMSMs, particularly at low speeds, remains a challenge. To overcome these limitations, this paper introduces a dual-mode regenerative braking control strategy using a cascaded buck-boost converter and fuzzy logic controller tailored for PMSM systems.

\section{System Architecture}
The proposed system consists of three core components:
\begin{itemize}
    \item A PMSM drive with field-oriented control (FOC)
    \item A cascaded bidirectional buck-boost converter interfaced with the traction battery
    \item A fuzzy logic controller (FLC) that determines control mode and converter duty cycle based on system conditions
\end{itemize}

The fuzzy controller receives inputs such as motor speed, SOC, brake pedal input, and converter current, and determines whether to engage voltage or current control mode. The converter adjusts energy flow direction and magnitude accordingly.

\section{Dual-Mode Control Strategy}
The dual-mode control logic operates as follows:
\begin{itemize}
    \item \textbf{Voltage Mode}: Activated at low speeds when back-EMF is insufficient to charge the battery directly. The converter steps up voltage to match battery requirements.
    \item \textbf{Current Mode}: Engaged at higher speeds where back-EMF exceeds battery voltage. Converter regulates braking current within battery safety limits.
\end{itemize}

The FLC continuously assesses system dynamics and switches modes in real time to optimize braking force and energy capture.

\section{Fuzzy Logic Controller Design}
The FLC is designed using Mamdani-type inference with three primary input variables:
\begin{itemize}
    \item Vehicle speed (Low, Medium, High)
    \item SOC (Low, Normal, High)
    \item Brake pressure (Light, Moderate, Heavy)
\end{itemize}

The output determines the duty cycle adjustment and mode selection. A rule base of 27 IF-THEN statements guides the decision-making process, providing adaptability to nonlinear driving conditions.

\section{Simulation and Results}
A simulation model was developed in MATLAB/Simulink incorporating the PMSM motor model, FOC, cascaded converter, and fuzzy logic controller. Various braking scenarios were simulated:
\begin{itemize}
    \item Urban deceleration (20 to 0 km/h)
    \item Emergency braking (80 to 0 km/h)
    \item Downhill braking with partial recovery
\end{itemize}

Results show:
\begin{itemize}
    \item Up to 35\% more energy recovered in low-speed scenarios versus traditional RB
    \item Stable battery current and SOC during aggressive braking
    \item Smooth braking torque with minimal jerk
\end{itemize}

\section{Conclusion}
The proposed dual-mode fuzzy-controlled regenerative braking strategy significantly enhances the efficiency and safety of EV braking systems using PMSMs. By dynamically switching between control modes, the system adapts to various speeds and battery states, ensuring maximum energy recovery and consistent performance. Future work will focus on hardware prototyping and experimental validation under real-world driving conditions.

\section*{Acknowledgments}
The authors would like to thank the academic and industrial contributors who provided technical insights and validation support for the simulation model.

\section*{References}
Add Elsevier-style references here once finalized
Example:


""End of Decument.
\end{document}